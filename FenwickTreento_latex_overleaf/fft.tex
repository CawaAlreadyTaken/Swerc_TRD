\section{FFT}

\subsection{FFT double}
\begin{itemize}
    \item Può avere problemi di approssimazione con numeri grandi, ricordarsi di arrotondare bene
    \item Non mi va con cose negative
    \item (Non me ne prendo responsabilità)
\end{itemize}
\begin{lstlisting}
  #include <bits/stdc++.h>

  using namespace std;

  #define _USE_MATH_DEFINES
  #include <complex>
  #include <vector>
  #include <cmath>

  std::vector<std::complex<double> > fast_fourier_transform(std::vector<std::complex<double> > x, bool inverse = false) {

      std::vector<std::complex<double> > w(x.size(), 0.0);
      w[0] = 1.0;
      for(int pow_2 = 1; pow_2 < (int)x.size(); pow_2 *= 2) {
          w[pow_2] = std::polar(1.0, 2*M_PI * pow_2/x.size() * (inverse ? 1 : -1) );
      }
      for(int i=3, last=2; i < (int)x.size(); i++) {
          if(w[i] == 0.0) {
              w[i] = w[last] * w[i-last];
          } else {
              last = i;
          }
      }

      for(int block_size = x.size(); block_size > 1; block_size /= 2) {
          std::vector<std::complex<double> > new_x(x.size());

          for(int start = 0; start < (int)x.size(); start += block_size) {
              for(int i=0; i<block_size; i++) {
                  new_x[start + block_size/2 * (i%2) + i/2] = x[start + i];
              }
          }
          x = new_x;
      }

      for(int block_size = 2; block_size <= (int)x.size(); block_size *= 2) {
          std::vector<std::complex<double> > new_x(x.size());
          int w_base_i = x.size() / block_size;

          for(int start = 0; start < (int)x.size(); start += block_size) {
              for(int i=0; i < block_size/2; i++) {
                  new_x[start+i]              = x[start+i] + w[w_base_i*i] * x[start + block_size/2 + i];
                  new_x[start+block_size/2+i] = x[start+i] - w[w_base_i*i] * x[start + block_size/2 + i];
              }
          }
          x = new_x;
      }
      return x;
  }
  
  struct Polynomial {
      std::vector<double> a;
      Polynomial(std::vector<double> new_a) : a(new_a) {}

      Polynomial operator*(Polynomial r) {
          int power_2 = 1;
          while(power_2 < (int)(a.size() + r.a.size() - 1)) {
              power_2 *= 2;
          }

          std::vector<std::complex<double> > x_l(power_2, 0.0);
          std::vector<std::complex<double> > x_r(power_2, 0.0);
          std::vector<std::complex<double> > product(power_2, 0.0);

          for(int i=0; i<(int)a.size(); i++) {
              x_l[i] = a[i];
          }
          for(int i=0; i<(int)r.a.size(); i++) {
              x_r[i] = r.a[i];
          }
          x_l = fast_fourier_transform(x_l);
          x_r = fast_fourier_transform(x_r);
          for(int i=0; i<power_2; i++) {
            product[i] = x_l[i] * x_r[i];
          }
          product = fast_fourier_transform(product, true);

          std::vector<double> result_a(a.size() + r.a.size() - 1);
          for(int i=0; i<(int)result_a.size(); i++) {
              result_a[i] = product[i].real() / power_2;
          }
          return result_a;
      }
  };

  int main() {
    vector<double> t(100000);
    for(int i=0; i<100000; i++) t[i]=i;
      Polynomial x_1(t);
      Polynomial x_2({2, 0, 1});
      Polynomial result = x_1 * x_2;

      ofstream out("output.txt");
      for(int i=0; i<result.a[i]; i++){
        out << (long long)(result.a[i] + 0.5 - (result.a[i]<0)) << " ";
      }
      return 0;
  }
\end{lstlisting}

\subsection{FFT modulo M}
\begin{itemize}
    \item Accetta numeri negativi, bisogna stare attenti ai moduli
    \item (Non me ne prendo responsabilità)
\end{itemize}
\begin{lstlisting}
 #include <bits/stdc++.h>
  using namespace std;

  #define N	100001
  #define L	18	/* L = ceil(log2(N * 2 - 1)) */
  #define N_	(1 << L)
  #define MD	469762049	/* MD = 56 * 2^23 + 1 */

  int *wu[L + 1], *wv[L + 1];

  int power(int a, int k) {
    long long b = a, p = 1;

    while (k) {
      if (k & 1)
        p = p * b % MD;
      b = b * b % MD;
      k >>= 1;
    }
    return p;
  }
  void init() {
    int l, i, u, v;
    u = power(3, (MD - 1) >> L);
    v = power(u, MD - 2);

    for (l = L; l > 0; l--) {
      int n = 1 << (l - 1);

      wu[l] = (int *) malloc(n * sizeof *wu[l]);
      wv[l] = (int *) malloc(n * sizeof *wv[l]);

      wu[l][0] = wv[l][0] = 1;
      for (i = 1; i < n; i++) {
        wu[l][i] = (long long) wu[l][i - 1] * u % MD;
        wv[l][i] = (long long) wv[l][i - 1] * v % MD;
      }

      u = (long long) u * u % MD, v = (long long) v * v % MD;
    }
  }
  void ntt_(int *aa, int l, int inverse) {
    if (l > 0) {
      int n = 1 << l;
      int m = n >> 1;
      int *ww = inverse ? wv[l] : wu[l];
      int i, j;
      ntt_(aa, l - 1, inverse);
      ntt_(aa + m, l - 1, inverse);
      for (i = 0; (j = i + m) < n; i++) {
        int a = aa[i];
        int b = (long long) aa[j] * ww[i] % MD;
        if ((aa[i] = a + b) >= MD)
          aa[i] -= MD;
        if ((aa[j] = a - b) < 0)
          aa[j] += MD;
      }
    }
  }
  void ntt(int *aa, int l, int inverse) {
    int n_ = 1 << l, i, j;
    for (i = 0, j = 1; j < n_; j++) {
      int b;
      int tmp;
      for (b = n_ >> 1; (i ^= b) < b; b >>= 1)
        ;
      if (i < j)
        tmp = aa[i], aa[i] = aa[j], aa[j] = tmp;
    }
    ntt_(aa, l, inverse);
  }
  void mult(int *aa, int n, int *bb, int m, int *out) {
    static int aa_[N_], bb_[N_];
    int l, n_, i, v;
    l = 0;
    while (1 << l <= n - 1 + m - 1)
      l++;
    n_ = 1 << l;
    memcpy(aa_, aa, n * sizeof *aa), memset(aa_ + n, 0, (n_ - n) * sizeof *aa_);
    memcpy(bb_, bb, m * sizeof *bb), memset(bb_ + m, 0, (n_ - m) * sizeof *bb_);
    ntt(aa_, l, 0), ntt(bb_, l,  0);
    for (i = 0; i < n_; i++)
      out[i] = (long long) aa_[i] * bb_[i] % MD;
    ntt(out, l, 1);
    v = power(n_, MD - 2);
    for (i = 0; i < n_; i++)
      out[i] = (long long) out[i] * v % MD;
  }
  int main() {
    static int aa[N], bb[N], out[N_];
    int n, m, i;
    init();
    scanf("%d%d", &n, &m), n++, m++;
    for (i = 0; i < n; i++)
      scanf("%d", &aa[i]);
    for (i = 0; i < m; i++)
      scanf("%d", &bb[i]);
    mult(aa, n, bb, m, out);
    for (i = 0; i < n + m - 1; i++)
      printf("%d ", out[i]);
      printf("\n");
    return 0;
  }
\end{lstlisting}

