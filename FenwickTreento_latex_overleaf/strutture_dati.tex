\section{Strutture Dati}

\subsection{Segment Tree base}
\begin{lstlisting}
int higherPowerOf2(int x) {
    int res = 1;
    while (res < x) res *= 2;
    return res;
}

struct SegmentTree {
    vector<int> data;
    SegmentTree(int n) : data(2 * higherPowerOf2(n), 0) {}

    int query(int i, int a, int b, int x, int y) {
        if (b <= x || a >= y) return 0;
        if (b <= y && a >= x) return data[i];

        return query(i*2,   a, (a+b)/2, x, y)
             + query(i*2+1, (a+b)/2, b, x, y);
    }

    int update(int i, int a, int b, int x, int v) {
        if (x < a || x >= b) return data[i];
        if (a == b-1) {
            assert(a == x);
            return data[i] = v;
        }

        return data[i] = update(i*2,   a, (a+b)/2, x, v)
                        + update(i*2+1, (a+b)/2, b, x, v);
    }

    int query(int x, int y) {
        assert(x <= y);
        return query(1, 0, data.size()/2, x, y);
    }

    void update(int x, int v) {
        update(1, 0, data.size()/2, x, v);
    }
};
\end{lstlisting}

\subsection{Segment Tree con Lazy Propagation}
\begin{lstlisting}
enum class Mode : char { none, add, set };
struct Node {
   ll min = numeric_limits<ll>::max();
   ll sum = 0;

   ll update = 0;
   Mode mode = Mode::none;
};
void setup(const vector<ll>& v, int a, int b, int i) {
   if (b-a == 1) {
      if (a < (ll)v.size()) {
         dat[i].min = v[a];
         dat[i].sum = v[a];
      }
      return;
   }
   setup(v, a, (a+b)/2, i*2);
   setup(v, (a+b)/2, b, i*2+1);
   setup(i);
}
void setup(int i) {
   if (i2 >= (ll)dat.size()) return;
   dat[i].min = min(dat[i*2].min, dat[i*2+1].min);
   dat[i].sum = dat[i*2].sum + dat[i*2+1].sum;
}
void lazyPropStep(int a, int b, int i, ll update, Mode mode) {
   if(mode == Mode::none) {
      return;
   } else if (mode == Mode::add) {
      if (dat[i].mode == Mode::none) {
         dat[i].update = 0; // just in case
      }
      dat[i].min += update;
      dat[i].sum += (b-a)*update;
      dat[i].update += update;
      if (dat[i].mode == Mode::none) {
         dat[i].mode = Mode::add; // do not change Mode::set
      }
   } else /* mode == Mode::set */ {
      dat[i].min = update;
      dat[i].sum = (b-a)*update;
      dat[i].update = update;
      dat[i].mode = Mode::set;
   }
}
void lazyProp(int a, int b, int i) {
   if (i*2 >= (ll)dat.size()) return;
   lazyPropStep(a, (a+b)/2, i*2,   dat[i].update, dat[i].mode);
   lazyPropStep((a+b)/2, b, i*2+1, dat[i].update, dat[i].mode);
   dat[i].update = 0;
   dat[i].mode = Mode::none;
}
ll queryMin(int l, int r, int a, int b, int i) {
   if (a>=r || b<=l) return numeric_limits<int>::max();
   if (a>=l && b<=r) return dat[i].min;
   lazyProp(a, b, i);
   return min(queryMin(l, r, a, (a+b)/2, i*2),
              queryMin(l, r, (a+b)/2, b, i*2+1));
}
ll querySum(int l, int r, int a, int b, int i) {
   if (a>=r || b<=l) return 0;
   if (a>=l && b<=r) return dat[i].sum;
   lazyProp(a, b, i);
   return querySum(l, r, a, (a+b)/2, i*2)
        + querySum(l, r, (a+b)/2, b, i*2+1);
}
void lazyAdd(int l, int r, ll x, int a, int b, int i) {
   if (a>=r || b<=l) return;
   lazyProp(a, b, i);
   if (a>=l && b<=r) {
      dat[i].min += x;
      dat[i].sum += (b-a)*x;
      dat[i].update = x;
      dat[i].mode = Mode::add;
      return;
   }
   lazyAdd(l, r, x, a, (a+b)/2, i*2);
   lazyAdd(l, r, x, (a+b)/2, b, i*2+1);
   setup(i);
}
void lazySet(int l, int r, ll x, int a, int b, int i) {
   if (a>=r || b<=l) return;
   lazyProp(a, b, i);
   if (a>=l && b<=r) {
      dat[i].min = x;
      dat[i].sum = (b-a)*x;
      dat[i].update = x;
      dat[i].mode = Mode::set;
      return;
   }
   lazySet(l, r, x, a, (a+b)/2, i*2);
   lazySet(l, r, x, (a+b)/2, b, i*2+1);
   setup(i);
}
\end{lstlisting}

\subsection{Fenwick Tree}
\begin{lstlisting}
int leastSignificantOneBit(int i){
	return i & (-i);
}

struct FenwickTree {
	vector<int> data;
	FenwickTree(int N) : data(N) {}
	void add(int pos, int value) {
		if (pos>=data.size()) return;
		data[pos] += value;
		add(pos + leastSignificantOneBit(pos), value);
	}
	int sumUpTo(int pos) {
		if (pos==0) return 0;
		return data[pos] + sumUpTo(pos - leastSignificantOneBit(pos));
	}
};
\end{lstlisting}

\subsection{Sparse Table}
\begin{itemize}
    \item Per ogni elemento di un array applico l'operazione ai range $[0,1), [0,2), [0,4), ...$ (potenze di 2) e salvo il valore in un array $st[N][32]$
    \item (vale per operazioni idempotenti, i.e. "a op a = a") per trovare il valore nel range $[l,r)$ in $O(1)$ basta trovare $k = max(k\_$ tali che $2^{k\_} \leq r-l$) e poi il risultato della query è "st[l][k] op st[r-($1LL<<k$)][k]"
\end{itemize}
Esempio: RMQ
\begin{lstlisting}
int st[K + 1][MAXN];

std::copy(array.begin(), array.end(), st[0]);

for (int i = 1; i <= K; i++)
    for (int j = 0; j + (1 << i) <= N; j++)
        st[i][j] = min(st[i - 1][j], st[i - 1][j + (1 << (i - 1))]);
        
// Precompute lg:
int lg[MAXN+1];
lg[1] = 0;
for (int i = 2; i <= MAXN; i++)
    lg[i] = lg[i/2] + 1;
// Query
int i = lg[R - L + 1];
int minimum = min(st[i][L], st[i][R - (1 << i) + 1]);
\end{lstlisting}
