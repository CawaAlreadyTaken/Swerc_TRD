\section{Grafi}

\subsection{Ordinamento Topologico (DAG)}
\begin{enumerate}
    \item Per ogni nodo esploro tutti i figli in DFS e dopo inserisco il risultato in uno stack
    \item Il risultato è lo stack ribaltato
\end{enumerate}

\subsection{SCC, Kosaraju, Strongly Connected Components (grafo diretto)}
\begin{enumerate}
    \item Calcolo l’ordinamento topologico
    \item Calcolo il grafo trasposto
    \item Esploro il grafo trasposto procedendo nell’ordine dell’ordinamento topologico. Ogni esplorazione è una componente fortemente connessa (eventualmente di 1 nodo).
\end{enumerate}
\begin{lstlisting}
struct Node {
   bool vis=false;
   int condensed=-1;
   vector<int> to, from;
};
struct CNode {
   vector<int> nodes;
   vector<int> to, from;
};

vector<int> ordinati;
function<void(int)> toposort = [&](int i){
    if(nodes[i].vis) return;
    nodes[i].vis=true;
    for(auto e:nodes[i].to) toposort(e);
    ordinati.push_back(i);
};
for(int n=0;n<N;++n) toposort(n);

int c=0;
vector<CNode> condensed;
function<void(int)> condense = [&](int i){
    if(nodes[i].condensed!=-1) {
        if(nodes[i].condensed!=c) {
            condensed[c].from.push_back(nodes[i].condensed);
            condensed[nodes[i].condensed].to.push_back(c);
        }
        return;
    }
    nodes[i].condensed=c;
    condensed[c].nodes.push_back(i);
    for(auto e:nodes[i].from) condense(e);
};

for(int n=0;n<N;++n){
  if(nodes[ordinati[N-n-1]].condensed==-1) {
     condensed.emplace_back();
     condense(ordinati[N-n-1]);
     ++c;
  }
}
\end{lstlisting}

\subsection{Max Flow}
\begin{enumerate}
    \item in BFS $\rightarrow$ O($V\cdot E^2$)
    \item in DFS $\rightarrow$ O($flow\cdot (V+E)$)
\end{enumerate}
\begin{lstlisting}
vector<int> parent(N);
auto bfsAugmentingPath = [&]() -> int {
   fill(parent.begin(), parent.end(), -1);
   parent[S] = -2; // prevent passing through S
   queue<pair<int, int>> q;
   q.push({S, numeric_limits<int>::max()});

   while (!q.empty()) {
       auto [i, flow] = q.front();
       q.pop();

       for (int e : adj[i]) {
          if (capacity[i][e] > 0 && parent[e] == -1) {
             parent[e] = i;
             if (e == T) return min(flow, capacity[i][e]);
             q.push({e, min(flow, capacity[i][e])});
          }
       }
   }
   return 0;
};

int flow=0;
while(1) {
    int partialFlow = bfsAugmentingPath();
    if (partialFlow == 0) break;
    flow += partialFlow;

    int last=T;
    while(last!=S){
       capacity[parent[last]][last] -= partialFlow;
       capacity[last][parent[last]] += partialFlow;
       last = parent[last];
    }
 }
\end{lstlisting}

\subsection{Tarjan, Articulation points and bridges (grafo non diretto)}
\begin{enumerate}
    \item Inizialmente settare $t=0$
    \item Fare DFS incrementando $t$ ogni volta che si attraversa un arco in avanti, cioè ogni volta che si vede un nuovo nodo
    \item Ogni nodo ha un $tEntrata$ e un $tMin$
    \begin{itemize}
        \item $tEntrata$ è il $t$ della prima volta in cui quel nodo è stato visto
        \item $tMin$ è il min tra $tEntrata$ e tutti i $tMin$ dei nodi adiacenti eccetto il padre nella DFS
    \end{itemize}
    \item Il nodo $b$ è un \textbf{articulation point} se esiste un nodo adiacente $a$ tale per cui $a.tMin \geq b.tEntrata$
    \item L'arco che connette due nodi $a$ e $b$ è un \textbf{bridge} se $a.tMin > b.tEntrata$
\end{enumerate}

\subsection{Bipartite Graph / Bicoloring}
\begin{lstlisting}
struct Node {
  int color=-1;
  vector<int> conn;
};

int32_t main() {
  vector<Node> nodes(N);
  queue<pair<int,int>> q;
  q.push({0,0});
  bool bicolorable=true;

  while (!q.empty()) {
    auto [i,c] = q.front();
    q.pop();

    if (nodes[i].color == -1) {
      nodes[i].color=c;
      for(auto&& con : nodes[i].conn) {
        q.push({con, (c+1)%2});
      }
    } else {
      if (nodes[i].color!=c) {
        bicolorable=false;
        break;
      }
    }
  }
}
\end{lstlisting}

\subsection{SPFA (Bellman-Ford's improved)}
\begin{itemize}
    \item Ricordarsi di aggiungere limite al numero di esecuzioni se possono esserci cicli negativi, altrimenti va all'infinito
\end{itemize}
\begin{lstlisting}
  const int INF = 1000000000;
  vector<vector<pair<int, int>>> adj;

  bool spfa(int s, vector<int>& d) {
      int n = adj.size();
      d.assign(n, INF);
      vector<int> cnt(n, 0);
      vector<bool> inqueue(n, false);
      queue<int> q;
      d[s] = 0;
      q.push(s);
      inqueue[s] = true;
      while (!q.empty()) {
          int v = q.front();
          q.pop();
          inqueue[v] = false;

          for (auto edge : adj[v]) {
              int to = edge.first;
              int len = edge.second;

              if (d[v] + len < d[to]) {
                  d[to] = d[v] + len;
                  if (!inqueue[to]) {
                      q.push(to);
                      inqueue[to] = true;
                      cnt[to]++;
                      if (cnt[to] > n)
                          return false;  // negative cycle
                  }
              }
          }
      }
      return true;
  }
\end{lstlisting}

