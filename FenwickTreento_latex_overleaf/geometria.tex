\section{Geometria}

\subsection{Vettori}
\begin{itemize}
    \item Prodotto vettore: $A\wedge B = A.x\cdot B.y - A.y\cdot B.x = |A|\cdot|B|\cdot sin$(angolo fra A e B)
    \begin{itemize}
        \item Proprietà: $A\wedge B = -B\wedge A, A\wedge A = 0, A\wedge {(B+C)} = A\wedge B + A\wedge C$
    \end{itemize}
    \item $(0,0)$, $A$ e $B$ allineati sse $A\wedge B = 0$
    \item Data retta orientata $AB$ e punto $C$, il prodotto $p=(A-C)\wedge {(B-C)}$ indica:
    \begin{itemize}
        \item che $C$ è: sulla retta sse $p=0$, a destra della retta sse $p<0$, a sinistra della retta sse $p>0$.
        \item che $A$, $B$ e $C$ sono: in ordine orario sse $p<0$, antiorario sse $p>0$
        \item \textbf{Area} triangolo $ABC = |p|/2$ (senza $/2$ per parallelogramma)
        \item \textbf{Distanza} di $C$ da $AB = |p|/(B-A)$
    \end{itemize}
    \item Area poligono $P_0, P_1, ..., P_n-1 = 1/2 \cdot |P_0 \wedge P_1 + P_1 \wedge P_2 + ... + P_{n-2} \wedge P_{n-1} + P_{n-1} \wedge P_0|$
    \item Per vedere se $P$ è dentro il poligono \textbf{convesso} $P_0, P_1, ..., P_n-1$: controllare se $P$ sempre dalla stessa parte di tutti i $P_0P_1, P_1P_2, ..., P_{n-1}P_0$
    \item Per vedere se $P$ è dentro un poligono \textbf{concavo}: controllare $\#$ intersezioni della semiretta $PQ$ con $Q$ scelto a caso molto grande: se $\#$ pari $P$ è esterno, se $\#$ dispari è interno
    \item $AB$ e $CD$ si intersecano sse ($C$ e $D$ da parti opposte di $AB$) e ($A$ e $B$ da parti opposte di $CD$)
    \begin{itemize}
        \item Intersezioni mantenute con trasformazioni lineari
    \end{itemize}
    \item \textbf{Ordinare} punti per \textbf{angolo}: $sort$ con $operator<(P,Q) = P \wedge Q < 0$
\end{itemize}

\subsection{Convex Hull}
\begin{enumerate}
    \item Trovare il punto più in basso (P0)
    \item Ordinare per angolo rispetto a P0 usando $(P-P0)\wedge(Q-P0)$, vedi sopra ($N\log N$)
    \item Andare avanti, e buttare in uno stack il punto che si trova
    \item Se l’angolo tra gli ultimi 3 è ottuso rimuovo l’elemento centrale dallo stack e ripeto
\end{enumerate}

\begin{lstlisting}
  struct pt {
      double x, y;
  };

  int orientation(pt a, pt b, pt c) {
      double v = a.x*(b.y-c.y)+b.x*(c.y-a.y)+c.x*(a.y-b.y);
      if (v < 0) return -1; // clockwise
      if (v > 0) return +1; // counter-clockwise
      return 0;
  }
  bool cw(pt a, pt b, pt c, bool include_collinear) {
      int o = orientation(a, b, c);
      return o < 0 || (include_collinear && o == 0);
  }
  bool collinear(pt a, pt b, pt c) { return orientation(a, b, c) == 0; }
  void convex_hull(vector<pt>& a, bool include_collinear = false) {
      pt p0 = *min_element(a.begin(), a.end(), [](pt a, pt b) {
          return make_pair(a.y, a.x) < make_pair(b.y, b.x);
      });
      sort(a.begin(), a.end(), [&p0](const pt& a, const pt& b) {
          int o = orientation(p0, a, b);
          if (o == 0)
              return (p0.x-a.x)*(p0.x-a.x) + (p0.y-a.y)*(p0.y-a.y)
                  < (p0.x-b.x)*(p0.x-b.x) + (p0.y-b.y)*(p0.y-b.y);
          return o < 0;
      });
      if (include_collinear) {
          int i = (int)a.size()-1;
          while (i >= 0 && collinear(p0, a[i], a.back())) i--;
          reverse(a.begin()+i+1, a.end());
      }

      vector<pt> st;
      for (int i = 0; i < (int)a.size(); i++) {
          while (st.size() > 1 && !cw(st[st.size()-2], st.back(), a[i], include_collinear))
              st.pop_back();
          st.push_back(a[i]);
      }
      a = st;
  }
\end{lstlisting}
