\section{Matematica}

\subsection{Fast exponentiation}
\begin{lstlisting}
int fastExp(int x, int e){
   if (e==0) return 1;
   int half = fastExp(x, e/2);
   return ((half*half % M) * (e%2 == 1 ? x : 1)) % M;
}
\end{lstlisting}

\subsection{Euclide esteso}
$A/B=d\; \& \;A\%B=C \rightarrow Bx+Cy = 1$ con $y=-x$ e $x = dx - y$

\subsection{Euclide esteso, ma iterativo}
\begin{lstlisting}
int gcd(int a, int b, int& x, int& y) {
    x = 1, y = 0;
    int x1 = 0, y1 = 1, a1 = a, b1 = b;
    while (b1) {
        int q = a1 / b1;
        tie(x, x1) = make_tuple(x1, x - q * x1);
        tie(y, y1) = make_tuple(y1, y - q * y1);
        tie(a1, b1) = make_tuple(b1, a1 - q * b1);
    }
    return a1;
}
\end{lstlisting}

\subsection{Fermat \& inverso moltiplicativo}
Inverso di $A = A^{(M-2)}\%M = fastExp(A, M-2)$

\subsection{Rabin Karp Hash}
\begin{lstlisting}
#define hash_t uint64_t
#define M 1000000007
#define P 59

hash_t getHash(const char* s, size_t l) {
   if (l==0) return 0;
   return (P*getHash(s+1, l-1) + s[0]) % M;
}

signed main() {
   array<int, 4002> Pexp;
   array<int, 4002> PexpMulInv;

   int p=1;
   for(int i=0;i<(int)Pexp.size();++i){
      Pexp[i] = p;
      PexpMulInv[i] = fastExp(p, M-2);
      p*=P; p%=M;
   }
   
   // calculate hashes for strings in S from 0 to any l
   vector<hash_t> hashes(N+1);
   int lasth=0;
   for(size_t l=0;l<N;++l){
      hashes[l]=lasth;
      lasth+=Pexp[l]*S[l];
      lasth%=M;
   }
   hashes[N]=lasth;

   // obtain the hash of s in range [n, n+l) with prefix sum
   hash_t hcmp = (((hashes[n + l] - hashes[n] + M) % M) * PexpMulInv[n]) % M;
}
\end{lstlisting}

\subsection{Gauss (solving system of linear equations)}
\begin{lstlisting}
const double EPS = 1e-9;
const int INF = 2; // no need for infinity

int gauss (vector < vector<double> > a, vector<double> & ans) {
    int n = (int) a.size();
    int m = (int) a[0].size() - 1;

    vector<int> where (m, -1);
    for (int col=0, row=0; col<m && row<n; ++col) {
        int sel = row;
        for (int i=row; i<n; ++i)
            if (abs (a[i][col]) > abs (a[sel][col]))
                sel = i;
        if (abs (a[sel][col]) < EPS)
            continue;
        for (int i=col; i<=m; ++i)
            swap (a[sel][i], a[row][i]);
        where[col] = row;

        for (int i=0; i<n; ++i)
            if (i != row) {
                double c = a[i][col] / a[row][col];
                for (int j=col; j<=m; ++j)
                    a[i][j] -= a[row][j] * c;
            }
        ++row;
    }

    ans.assign (m, 0);
    for (int i=0; i<m; ++i)
        if (where[i] != -1)
            ans[i] = a[where[i]][m] / a[where[i]][i];
    for (int i=0; i<n; ++i) {
        double sum = 0;
        for (int j=0; j<m; ++j)
            sum += ans[j] * a[i][j];
        if (abs (sum - a[i][m]) > EPS)
            return 0;
    }

    for (int i=0; i<m; ++i)
        if (where[i] == -1)
            return INF;
    return 1;
}
\end{lstlisting}

\subsection{Matexp}
\begin{lstlisting}
#include <bits/stdc++.h>
using namespace std;

const int N = 3;

const long long M = 1000000007;

void multiply (long long A[N][N], long long B[N][N]){
    long long R[N][N];

    for (int i = 0; i < N; i++){
        for (int j = 0; j < N; j++){
            R[i][j] = 0;
            for (int k = 0; k < N; k++){
                R[i][j] = (R[i][j] + A[i][k] * B[k][j]) % M;
            }
        }
    }

    for (int i = 0; i < N; i++){
        for (int j = 0; j < N; j++){
            A[i][j] = R[i][j];
        }
    }
}

void power_matrix (long long A[N][N], int n){
    long long B[N][N];

    for (int i = 0; i < N; i++){
        for (int j = 0; j < N; j++){
            B[i][j] = A[i][j];
        }
    }

    n = n - 1;
    while (n > 0)
    {
        if (n & 1)
            multiply (A, B);

        multiply (B,B);

        n = n >> 1;  
    }
}

long long solve_recurrence (long long A[N][N], long long B[N][1], int n){
    if (n < N)
        return B[N - 1 - n][0];
    
    power_matrix (A, n - N + 1);
    
    long long result = 0;
    
    for (int i = 0; i < N; i++)
        result = (result + A[0][i] * B[i][0]) % M;
    
    return result;
}

int main ()
{

    long long A[N][N] = {{2, 1, 3}, {1, 0, 0}, {0, 1, 0}};
    long long B[N][1] = {{3}, {2}, {1}};
    
    int n = 5;
    
    long long R_n = solve_recurrence (A, B, n);
    
    cout << "R_" << n << " = " << R_n; 

    return 0;
}
\end{lstlisting}

