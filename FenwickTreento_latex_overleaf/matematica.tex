\section{Matematica}

\subsection{Fast exponentiation}
\begin{lstlisting}
int fastExp(int x, int e){
   if (e==0) return 1;
   int half = fastExp(x, e/2);
   return ((half*half % M) * (e%2 == 1 ? x : 1)) % M;
}
\end{lstlisting}

\subsection{Euclide esteso}
$A/B=d\; \& \;A\%B=C \rightarrow Bx+Cy = 1$ con $y=-x$ e $x = dx - y$

\subsection{Fermat \& inverso moltiplicativo}
Inverso di $A = A^{(M-2)}\%M = fastExp(A, M-2)$

\subsection{Rabin Karp Hash}
\begin{lstlisting}
#define hash_t uint64_t
#define M 1000000007
#define P 59

hash_t getHash(const char* s, size_t l) {
   if (l==0) return 0;
   return (P*getHash(s+1, l-1) + s[0]) % M;
}

signed main() {
   array<int, 4002> Pexp;
   array<int, 4002> PexpMulInv;

   int p=1;
   for(int i=0;i<(int)Pexp.size();++i){
      Pexp[i] = p;
      PexpMulInv[i] = fastExp(p, M-2);
      p*=P; p%=M;
   }
   
   // calculate hashes for strings in S from 0 to any l
   vector<hash_t> hashes(N+1);
   int lasth=0;
   for(size_t l=0;l<N;++l){
      hashes[l]=lasth;
      lasth+=Pexp[l]*S[l];
      lasth%=M;
   }
   hashes[N]=lasth;

   // obtain the hash of s in range [n, n+l) with prefix sum
   hash_t hcmp = (((hashes[n + l] - hashes[n] + M) % M) * PexpMulInv[n]) % M;
}
\end{lstlisting}
