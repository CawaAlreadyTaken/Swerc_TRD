\documentclass{article}
\usepackage[margin=1.5cm]{geometry}
\usepackage{listings}
\usepackage{setspace}
\setstretch{0}
\usepackage{titlesec}
\titlespacing\section{0pt}{12pt plus 4pt minus 2pt}{6pt plus 2pt minus 2pt}
\titlespacing\subsection{0pt}{10pt plus 4pt minus 2pt}{4pt plus 2pt minus 2pt}
\usepackage{xcolor}
\lstset{language=C++, 
        basicstyle=\ttfamily\small, 
        keywordstyle=\color{blue},
        commentstyle=\color{green},
        morecomment=[l][\color{magenta}]{\#},
        breaklines=true,
        postbreak=\mbox{\textcolor{red}{$\hookrightarrow$}\space}}

\usepackage{fancyhdr}
\pagestyle{fancy}
\fancyhf{} % pulisce l'intestazione e il piè di pagina predefiniti
\rhead{Fenwick Treento} % intestazione a destra
% \lhead{Nome della Squadra} % intestazione a sinistra
\cfoot{\thepage} % numero di pagina al centro del piè di pagina

\begin{document}

\section{Grafi}

\subsection{Ordinamento Topologico (DAG)}
\begin{enumerate}
    \item Per ogni nodo esploro tutti i figli e dopo inserisco il risultato in uno stack
    \item Il risultato è lo stack ribaltato
\end{enumerate}

\subsection{SCC, Kosaraju, Strongly Connected Components (grafo diretto)}
\begin{enumerate}
    \item Calcolo l’ordinamento topologico
    \item Calcolo il grafo trasposto
    \item Esploro il grafo trasposto procedendo nell’ordine dell’ordinamento topologico. Ogni esplorazione è una componente fortemente connessa (eventualmente di 1 nodo).
\end{enumerate}
\begin{lstlisting}
struct Node {
   bool vis=false;
   int condensed=-1;
   vector<int> to, from;
};
struct CNode {
   vector<int> nodes;
   vector<int> to, from;
};

vector<int> ordinati;
function<void(int)> toposort = [&](int i){
    if(nodes[i].vis) return;
    nodes[i].vis=true;
    for(auto e:nodes[i].to) toposort(e);
    ordinati.push_back(i);
};
for(int n=0;n<N;++n) toposort(n);

int c=0;
vector<CNode> condensed;
function<void(int)> condense = [&](int i){
    if(nodes[i].condensed!=-1) {
        if(nodes[i].condensed!=c) {
            condensed[c].from.push_back(nodes[i].condensed);
            condensed[nodes[i].condensed].to.push_back(c);
        }
        return;
    }
    nodes[i].condensed=c;
    condensed[c].nodes.push_back(i);
    for(auto e:nodes[i].from) condense(e);
};

for(int n=0;n<N;++n){
  if(nodes[ordinati[N-n-1]].condensed==-1) {
     condensed.emplace_back();
     condense(ordinati[N-n-1]);
     ++c;
  }
}
\end{lstlisting}

\subsection{Max Flow}
\begin{enumerate}
    \item in BFS $\rightarrow$ O($V\cdot E^2$)
    \item in DFS $\rightarrow$ O($flow\cdot (V+E)$)
\end{enumerate}
\begin{lstlisting}
vector<int> parent(N);
auto bfsAugmentingPath = [&]() -> int {
   fill(parent.begin(), parent.end(), -1);
   parent[S] = -2; // prevent passing through S
   queue<pair<int, int>> q;
   q.push({S, numeric_limits<int>::max()});

   while (!q.empty()) {
       auto [i, flow] = q.front();
       q.pop();

       for (int e : adj[i]) {
          if (capacity[i][e] > 0 && parent[e] == -1) {
             parent[e] = i;
             if (e == T) return min(flow, capacity[i][e]);
             q.push({e, min(flow, capacity[i][e])});
          }
       }
   }
   return 0;
};

int flow=0;
while(1) {
    int partialFlow = bfsAugmentingPath();
    if (partialFlow == 0) break;
    flow += partialFlow;

    int last=T;
    while(last!=S){
       capacity[parent[last]][last] -= partialFlow;
       capacity[last][parent[last]] += partialFlow;
       last = parent[last];
    }
 }
\end{lstlisting}

\subsection{Tarjan, Articulation points and bridges (grafo non diretto)}
\begin{enumerate}
    \item Inizialmente settare $t=0$
    \item Fare DFS incrementando $t$ ogni volta che si attraversa un arco in avanti, cioè ogni volta che si vede un nuovo nodo
    \item Ogni nodo ha un $tEntrata$ e un $tMin$
    \begin{itemize}
        \item $tEntrata$ è il $t$ della prima volta in cui quel nodo è stato visto
        \item $tMin$ è il min tra $tEntrata$ e tutti i $tMin$ dei nodi adiacenti eccetto il padre nella DFS
    \end{itemize}
    \item Il nodo $b$ è un \textbf{articulation point} se esiste un nodo adiacente $a$ tale per cui $a.tMin \geq b.tEntrata$
    \item L'arco che connette due nodi $a$ e $b$ è un \textbf{bridge} se $a.tMin > b.tEntrata$
\end{enumerate}

\subsection{Bipartite Graph / Bicoloring}
\begin{lstlisting}
struct Node {
  int color=-1;
  vector<int> conn;
};

int32_t main() {
  vector<Node> nodes(N);
  queue<pair<int,int>> q;
  q.push({0,0});
  bool bicolorable=true;

  while (!q.empty()) {
    auto [i,c] = q.front();
    q.pop();

    if (nodes[i].color == -1) {
      nodes[i].color=c;
      for(auto&& con : nodes[i].conn) {
        q.push({con, (c+1)%2});
      }
    } else {
      if (nodes[i].color!=c) {
        bicolorable=false;
        break;
      }
    }
  }
}
\end{lstlisting}

\section{Alberi}

\subsection{UFDS, Union Find Disjoint Set}
Si può ottimizzare ammortizzando di più mettendo in parent():
\begin{lstlisting}
    return nodes[i].parent = parent(nodes[i].parent);
\end{lstlisting}
Codice:
\begin{lstlisting}
struct Node {
    int parent=-1, rank=0;
};
function<int(int)> parent = [&](int i) {
    if (nodes[i].parent == -1) return i;
    return parent(nodes[i].parent);
};
auto connect = [&](int a, int b) {
    int pa = parent(a);
    int pb = parent(b);
    if (pa==pb) return false;
    if (nodes[pa].depth > nodes[pb].depth) swap(pa, pb);
    nodes[pa].parent = pb;
    nodes[pb].depth = max(nodes[pb].depth, 1+nodes[pa].depth);
    return true;
};
\end{lstlisting}

\subsection{LCA, Lowest Common Ancestor}
\begin{enumerate}
    \item Salvarsi per ogni nodo la profondità dalla radice
    \item Trovare con binary lifting per ogni nodo l’array "antenati[20]" (e se serve anche anche "dist[20]" o "minarco[20]") dove "antenati[e]" indica l’antenato risalendo di $2^e$ nodi. Basta prima impostare gli "antenati[0]" e poi fare un "for($0<e<20$) for(nodo in albero) nodo.antenati[e] = albero[nodo.antenati[e-1]].antenati[e-1]"
\end{enumerate}
\begin{lstlisting}
for(int e=1;e<20;++e) {
   for(int i=0; i<N; i++) {
      int half = albero[i].antenati[e-1];
      albero[i].antenati[e] = albero[half].antenati[e-1];
      albero[i].dist[e] = albero[i].dist[e-1] + albero[half].dist[e-1];
      albero[i].minarco[e] = min(albero[i].minarco[e-1], albero[half].minarco[e-1]);
   }
}


int lift(int v, int h) {
   for(int e=20; e>=0; e--) {
      if(h & (1<<e)) {
         v = albero[v].antenati[e];
      }
   }
   return v;
}

int lca(int u, int v) {
   int hu=albero[u].altezza, hv=albero[v].altezza;
   if (hu>hv) {
      u=lift(u, hu-hv);
   } else if (hv>hu) {
      v=lift(v, hv-hu);
   }

   if(u==v) {
      return u;
   }

   for(int e=19; e>=0; e--) {
      if(albero[u].antenati[e]!=albero[v].antenati[e]) {
         u=albero[u].antenati[e];
         v=albero[v].antenati[e];
      }
   }
   return albero[u].antenati[0];
}
\end{lstlisting}
Oppure si può fare anche in $O(n)$:
\begin{enumerate}
    \item dfs dalla radice salvando quando nodi vengono aperti e chiusi in array
    \item fare Range Minimum Query con una Sparse Table (la costruzione richiede $O(n\cdot \log n)$)
\end{enumerate}

\subsection{MST, Minimum Spanning Tree}
\begin{enumerate}
    \item Sortare gli archi per il peso ($O(E\cdot \log E)=O(V^2\cdot \log V)$)
    \item Partire dagli archi più piccoli ed aggiungerli all'albero, ma solo se questo non lo rende non più un albero ($O(E)=O(V^2)$)
    \item Usare Union Find per capire se un arco unirebbe due nodi già collegati e romperebbe l'albero
\end{enumerate}
\begin{itemize}
    \item Se serve trovare il \textbf{maximum} spanning tree basta scegliere gli archi più grossi invece che più piccoli
    \item Se serve trovare il \textbf{secondo} minimum spanning tree, si può:
    \begin{itemize}
        \item Per ogni percorso che connette due nodi nel MST, trovare l'arco massimo nel percorso con $V$ DFS ($O(V^2)$)
        \item Per ogni arco $a-b$ che non è già nel MST, calcolare di quanto aumenterebbe il peso totale del MST se si aggiungesse quell'arco e si togliesse però l'arco massimo nel percorso $a-b$
        \item Il minimo dei pesi totali trovati sopra corrisponde al second minimum spanning tree
    \end{itemize}
\end{itemize}

\section{Strutture Dati}

\subsection{Segment Tree base}
\begin{lstlisting}
int higherPowerOf2(int x) {
    int res = 1;
    while (res < x) res *= 2;
    return res;
}

struct SegmentTree {
    vector<int> data;
    SegmentTree(int n) : data(2 * higherPowerOf2(n), 0) {}

    int query(int i, int a, int b, int x, int y) {
        if (b <= x || a >= y) return 0;
        if (b <= y && a >= x) return data[i];

        return query(i*2,   a, (a+b)/2, x, y)
             + query(i*2+1, (a+b)/2, b, x, y);
    }

    int update(int i, int a, int b, int x, int v) {
        if (x < a || x >= b) return data[i];
        if (a == b-1) {
            assert(a == x);
            return data[i] = v;
        }

        return data[i] = update(i*2,   a, (a+b)/2, x, v)
                        + update(i*2+1, (a+b)/2, b, x, v);
    }

    int query(int x, int y) {
        assert(x <= y);
        return query(1, 0, data.size()/2, x, y);
    }

    void update(int x, int v) {
        update(1, 0, data.size()/2, x, v);
    }
};
\end{lstlisting}

\subsection{Segment Tree con Lazy Propagation}
\begin{lstlisting}
enum class Mode : char { none, add, set };
struct Node {
   ll min = numeric_limits<ll>::max();
   ll sum = 0;

   ll update = 0;
   Mode mode = Mode::none;
};
void setup(const vector<ll>& v, int a, int b, int i) {
   if (b-a == 1) {
      if (a < (ll)v.size()) {
         dat[i].min = v[a];
         dat[i].sum = v[a];
      }
      return;
   }
   setup(v, a, (a+b)/2, i*2);
   setup(v, (a+b)/2, b, i*2+1);
   setup(i);
}
void setup(int i) {
   if (i2 >= (ll)dat.size()) return;
   dat[i].min = min(dat[i*2].min, dat[i*2+1].min);
   dat[i].sum = dat[i*2].sum + dat[i*2+1].sum;
}
void lazyPropStep(int a, int b, int i, ll update, Mode mode) {
   if(mode == Mode::none) {
      return;
   } else if (mode == Mode::add) {
      if (dat[i].mode == Mode::none) {
         dat[i].update = 0; // just in case
      }
      dat[i].min += update;
      dat[i].sum += (b-a)*update;
      dat[i].update += update;
      if (dat[i].mode == Mode::none) {
         dat[i].mode = Mode::add; // do not change Mode::set
      }
   } else /* mode == Mode::set */ {
      dat[i].min = update;
      dat[i].sum = (b-a)*update;
      dat[i].update = update;
      dat[i].mode = Mode::set;
   }
}
void lazyProp(int a, int b, int i) {
   if (i*2 >= (ll)dat.size()) return;
   lazyPropStep(a, (a+b)/2, i*2,   dat[i].update, dat[i].mode);
   lazyPropStep((a+b)/2, b, i*2+1, dat[i].update, dat[i].mode);
   dat[i].update = 0;
   dat[i].mode = Mode::none;
}
ll queryMin(int l, int r, int a, int b, int i) {
   if (a>=r || b<=l) return numeric_limits<int>::max();
   if (a>=l && b<=r) return dat[i].min;
   lazyProp(a, b, i);
   return min(queryMin(l, r, a, (a+b)/2, i*2),
              queryMin(l, r, (a+b)/2, b, i*2+1));
}
ll querySum(int l, int r, int a, int b, int i) {
   if (a>=r || b<=l) return 0;
   if (a>=l && b<=r) return dat[i].sum;
   lazyProp(a, b, i);
   return querySum(l, r, a, (a+b)/2, i*2)
        + querySum(l, r, (a+b)/2, b, i*2+1);
}
void lazyAdd(int l, int r, ll x, int a, int b, int i) {
   if (a>=r || b<=l) return;
   lazyProp(a, b, i);
   if (a>=l && b<=r) {
      dat[i].min += x;
      dat[i].sum += (b-a)*x;
      dat[i].update = x;
      dat[i].mode = Mode::add;
      return;
   }
   lazyAdd(l, r, x, a, (a+b)/2, i*2);
   lazyAdd(l, r, x, (a+b)/2, b, i*2+1);
   setup(i);
}
void lazySet(int l, int r, ll x, int a, int b, int i) {
   if (a>=r || b<=l) return;
   lazyProp(a, b, i);
   if (a>=l && b<=r) {
      dat[i].min = x;
      dat[i].sum = (b-a)*x;
      dat[i].update = x;
      dat[i].mode = Mode::set;
      return;
   }
   lazySet(l, r, x, a, (a+b)/2, i*2);
   lazySet(l, r, x, (a+b)/2, b, i*2+1);
   setup(i);
}
\end{lstlisting}

\subsection{Fenwick Tree}
\begin{lstlisting}
int leastSignificantOneBit(int i){
	return i & (-i);
}

struct FenwickTree {
	vector<int> data;
	FenwickTree(int N) : data(N) {}
	void add(int pos, int value) {
		if (pos>=data.size()) return;
		data[pos] += value;
		add(pos + leastSignificantOneBit(pos), value);
	}
	int sumUpTo(int pos) {
		if (pos==0) return 0;
		return data[pos] + sumUpTo(pos - leastSignificantOneBit(pos));
	}
};
\end{lstlisting}

\subsection{Sparse Table}
\begin{itemize}
    \item Per ogni elemento di un array applico l'operazione ai range $[0,1), [0,2), [0,4), ...$ (potenze di 2) e salvo il valore in un array $st[N][32]$
    \item (vale per operazioni idempotenti, i.e. "a op a = a") per trovare il valore nel range $[l,r)$ in $O(1)$ basta trovare $k = max(k\_$ tali che $2^{k\_} \leq r-l$) e poi il risultato della query è "st[l][k] op st[r-($1LL<<k$)][k]"
\end{itemize}

\section{Algoritmi vari}

\subsection{Zaino / 0-1 knapsack}
\begin{enumerate}
    \item In input ci sono il numero di oggetti $N$, la capienza dello zaino $C$, i pesi $W[N]$ e i valori $V[N]$
    \item Caso base $f(N,c) = 0$
    \item Caso ricorsivo $f(i,c) = (W[i] \leq c$ ? $max(f(i+1, c), f(i+1, c-W[i]) + V[i]) : f(i+1, c))$
    \item Risultato è $f(0,C)$
    \item Usare $mem[i][c]$ per salvare risultati e fare DP
\end{enumerate}

\subsection{LIS, Longest Increasing Subsequence}
\begin{lstlisting}
int main() {
    int N;
    cin>>n;
    vector<int> pesi(N), ultimoPreso(N+1, numeric_limits<int>::max());
    ultimoPreso[0]=0;

    for(int i=0; i<N; i++) cin>>pesi[i];

    for(auto p : pesi) {
        auto it = lower_bound(ultimoPreso.begin(), ultimoPreso.end(), p);
        *it = p;
    }

    auto it = lower_bound(ultimoPreso.begin(), ultimoPreso.end(), numeric_limits<int>::max());
    cout << (it - ultimoPreso.begin() - 1);
}
\end{lstlisting}

\section{Matematica}

\subsection{Fast exponentiation}
\begin{lstlisting}
int fastExp(int x, int e){
   if (e==0) return 1;
   int half = fastExp(x, e/2);
   return ((half*half % M) * (e%2 == 1 ? x : 1)) % M;
}
\end{lstlisting}

\subsection{Euclide esteso}
$A/B=d\; \& \;A\%B=C \rightarrow Bx+Cy = 1$ con $y=-x$ e $x = dx - y$

\subsection{Fermat \& inverso moltiplicativo}
Inverso di $A = A^{(M-2)}\%M = fastExp(A, M-2)$

\subsection{Rabin Karp Hash}
\begin{lstlisting}
#define hash_t uint64_t
#define M 1000000007
#define P 59

hash_t getHash(const char* s, size_t l) {
   if (l==0) return 0;
   return (P*getHash(s+1, l-1) + s[0]) % M;
}

signed main() {
   array<int, 4002> Pexp;
   array<int, 4002> PexpMulInv;

   int p=1;
   for(int i=0;i<(int)Pexp.size();++i){
      Pexp[i] = p;
      PexpMulInv[i] = fastExp(p, M-2);
      p*=P; p%=M;
   }
   
   // calculate hashes for strings in S from 0 to any l
   vector<hash_t> hashes(N+1);
   int lasth=0;
   for(size_t l=0;l<N;++l){
      hashes[l]=lasth;
      lasth+=Pexp[l]*S[l];
      lasth%=M;
   }
   hashes[N]=lasth;

   // obtain the hash of s in range [n, n+l) with prefix sum
   hash_t hcmp = (((hashes[n + l] - hashes[n] + M) % M) * PexpMulInv[n]) % M;
}
\end{lstlisting}

\section{Geometria}

\subsection{Vettori}
\begin{itemize}
    \item Prodotto vettore: $A^B = A.x\cdot B.y - A.y\cdot B.x = |A|\cdot|B|\cdot sin$(angolo fra A e B)
    \begin{itemize}
        \item Proprietà: $A^B = -B^A, A^A = 0, A^{(B+C)} = A^B + A^C$
    \end{itemize}
    \item $(0,0)$, $A$ e $B$ allineati sse $A^B = 0$
    \item Data retta orientata $AB$ e punto $C$, il prodotto $p=(A-C)^{(B-C)}$ indica:
    \begin{itemize}
        \item che $C$ è: sulla retta sse $p=0$, a destra della retta sse $p<0$, a sinistra della retta sse $p>0$.
        \item che $A$, $B$ e $C$ sono: in ordine orario sse $p<0$, antiorario sse $p>0$
        \item \textbf{Area} triangolo $ABC = |p|/2$ (senza $/2$ per parallelogramma)
        \item \textbf{Distanza} di $C$ da $AB = |p|/(B-A)$
    \end{itemize}
    \item Area poligono $P_0, P_1, ..., P_n-1 = 1/2 \cdot |P_0 \wedge P_1 + P_1 \wedge P_2 + ... + P_{n-2} \wedge P_{n-1} + P_{n-1} \wedge P_0|$
    \item Per vedere se $P$ è dentro il poligono \textbf{convesso} $P_0, P_1, ..., P_n-1$: controllare se $P$ sempre dalla stessa parte di tutti i $P_0P_1, P_1P_2, ..., P_{n-1}P_0$
    \item Per vedere se $P$ è dentro un poligono \textbf{concavo}: controllare $\#$ intersezioni della semiretta $PQ$ con $Q$ scelto a caso molto grande: se $\#$ pari $P$ è esterno, se $\#$ dispari è interno
    \item $AB$ e $CD$ si intersecano sse ($C$ e $D$ da parti opposte di $AB$) e ($A$ e $B$ da parti opposte di $CD$)
    \begin{itemize}
        \item Intersezioni mantenute con trasformazioni lineari
    \end{itemize}
    \item \textbf{Ordinare} punti per \textbf{angolo}: $sort$ con $operator<(P,Q) = P \wedge Q < 0$
\end{itemize}

\subsection{Convex Hull}
\begin{enumerate}
    \item Trovare il punto più in basso (P0)
    \item Ordinare per angolo rispetto a P0 usando $(P-P0)\wedge(Q-P0)$, vedi sopra ($N\log N$)
    \item Andare avanti, e buttare in uno stack il punto che si trova
    \item Se l’angolo tra gli ultimi 3 è ottuso rimuovo l’elemento centrale dallo stack e ripeto
\end{enumerate}

\section{Utilities}

\subsection{Date e tempo}
\begin{lstlisting}
struct std::tm tmp;

ll readDate(){
	cin>>get_time(&tmp, "%Y-%m-%d %H:%M");
	return chrono::system_clock::from_time_t(mktime(&tmp)).time_since_epoch().count();
}
\end{lstlisting}

\subsection{STL}
\begin{itemize}
    \item operator$<$ Deve ritornare $false$ in caso di uguaglianza!
    \item \begin{lstlisting}
        priority_queue<pair<int,int>, vector<pair<int,int>>, greater<pair<int, int>>>
    \end{lstlisting}
    \item auto cmp = [](const T\& a, const T\& b)\{ return /* ... */; \};
    \item $priority\_queue<T,vec<T>,decltype(cmp)>$ pq\{cmp\};
\end{itemize}

\section{In caso di Errore}

\subsection{Wrong answer WA}
\begin{enumerate}
    \item $\#$define int long long
    \item $1LL$ invece che $1$
    \item Se ci sono moduli, vengono fatti dappertutto?
    \item Stampa la tua soluzione!
    \item Stai cancellando tutte le strutture di dati tra i casi di test?
    \item Il tuo algoritmo può gestire l'intera gamma di input?
    \item Leggi di nuovo l'intero testo del problema.
    \item Il tuo formato di output è corretto? (inclusi gli spazi bianchi)
    \item Gestisci correttamente tutti i corner case?
    \item Hai compreso correttamente il problema?
    \item Sei sicuro che il tuo algoritmo funzioni?
    \item A quali casi speciali non hai pensato?
    \item Sei sicuro che le funzioni STL che usi funzionino come pensi?
    \item Crea alcuni casi di prova su cui far girare il tuo algoritmo.
    \item Esegui l'algoritmo per un caso semplice.
    \item Ripassa questa lista.
    \item Spiega il tuo algoritmo ad un compagno di squadra.
    \item Chiedi al compagno di squadra di guardare il tuo codice.
    \item Vai a fare una piccola passeggiata, per esempio al bagno.
    \item Riscrivi la tua soluzione dall'inizio o fallo fare ad un compagno di squadra.
\end{enumerate}

\subsection{Runtime error RE}
\begin{enumerate}
    \item Hai testato tutti i corner case localmente?
    \item Stai accedendo ad indici out of bound di qualche vettore? (usa $.at()$)
    \item Qualche possibile divisione per 0? (mod 0 per esempio)
    \item Qualsiasi possibile ricorsione infinita?
\end{enumerate}

\subsection{Time limit exceeded TLE}
\begin{enumerate}
    \item $\#$pragma GCC optimize("O3")
    \item Hai qualche possibile loop infinito?
    \item Qual è la complessità del tuo algoritmo?
    \item Stai copiando molti dati non necessari? (usa le reference)
    \item Quanto è grande l'input e l'output?
    \item Cosa pensano i tuoi compagni di squadra del tuo algoritmo?
\end{enumerate}

\subsection{Memory limit exceeded MLE}
\begin{enumerate}
    \item Qual è la quantità massima di memoria di cui il vostro algoritmo dovrebbe avere bisogno?
    \item Ci sono memory leak? (usa la STL invece)
    \item Stai cancellando tutte le strutture di dati tra un test e l'altro?
\end{enumerate}

\subsection{Output limit exceeded OLE}
\begin{enumerate}
    \item Avete rimosso tutte le stampe di debug?
    \item Il vostro ciclo di output può andare in tilt?
\end{enumerate}

\section{Aggiunte Matte}

\subsection{FFT double}
\begin{itemize}
    \item Può avere problemi di approssimazione con numeri grandi, ricordarsi di arrotondare bene
    \item Non mi va con cose negative
    \item (Non me ne prendo responsabilità)
\end{itemize}
\begin{lstlisting}
  #include <bits/stdc++.h>

  using namespace std;

  #define _USE_MATH_DEFINES
  #include <complex>
  #include <vector>
  #include <cmath>

  std::vector<std::complex<double> > fast_fourier_transform(std::vector<std::complex<double> > x, bool inverse = false) {

      std::vector<std::complex<double> > w(x.size(), 0.0);
      w[0] = 1.0;
      for(int pow_2 = 1; pow_2 < (int)x.size(); pow_2 *= 2) {
          w[pow_2] = std::polar(1.0, 2*M_PI * pow_2/x.size() * (inverse ? 1 : -1) );
      }
      for(int i=3, last=2; i < (int)x.size(); i++) {
          if(w[i] == 0.0) {
              w[i] = w[last] * w[i-last];
          } else {
              last = i;
          }
      }

      for(int block_size = x.size(); block_size > 1; block_size /= 2) {
          std::vector<std::complex<double> > new_x(x.size());

          for(int start = 0; start < (int)x.size(); start += block_size) {
              for(int i=0; i<block_size; i++) {
                  new_x[start + block_size/2 * (i%2) + i/2] = x[start + i];
              }
          }
          x = new_x;
      }

      for(int block_size = 2; block_size <= (int)x.size(); block_size *= 2) {
          std::vector<std::complex<double> > new_x(x.size());
          int w_base_i = x.size() / block_size;

          for(int start = 0; start < (int)x.size(); start += block_size) {
              for(int i=0; i < block_size/2; i++) {
                  new_x[start+i]              = x[start+i] + w[w_base_i*i] * x[start + block_size/2 + i];
                  new_x[start+block_size/2+i] = x[start+i] - w[w_base_i*i] * x[start + block_size/2 + i];
              }
          }
          x = new_x;
      }
      return x;
  }
  
  struct Polynomial {
      std::vector<double> a;
      Polynomial(std::vector<double> new_a) : a(new_a) {}

      Polynomial operator*(Polynomial r) {
          int power_2 = 1;
          while(power_2 < (int)(a.size() + r.a.size() - 1)) {
              power_2 *= 2;
          }

          std::vector<std::complex<double> > x_l(power_2, 0.0);
          std::vector<std::complex<double> > x_r(power_2, 0.0);
          std::vector<std::complex<double> > product(power_2, 0.0);

          for(int i=0; i<(int)a.size(); i++) {
              x_l[i] = a[i];
          }
          for(int i=0; i<(int)r.a.size(); i++) {
              x_r[i] = r.a[i];
          }
          x_l = fast_fourier_transform(x_l);
          x_r = fast_fourier_transform(x_r);
          for(int i=0; i<power_2; i++) {
            product[i] = x_l[i] * x_r[i];
          }
          product = fast_fourier_transform(product, true);

          std::vector<double> result_a(a.size() + r.a.size() - 1);
          for(int i=0; i<(int)result_a.size(); i++) {
              result_a[i] = product[i].real() / power_2;
          }
          return result_a;
      }
  };

  int main() {
    vector<double> t(100000);
    for(int i=0; i<100000; i++) t[i]=i;
      Polynomial x_1(t);
      Polynomial x_2({2, 0, 1});
      Polynomial result = x_1 * x_2;

      ofstream out("output.txt");
      for(int i=0; i<result.a[i]; i++){
        out << (long long)(result.a[i] + 0.5 - (result.a[i]<0)) << " ";
      }
      return 0;
  }
\end{lstlisting}

\subsection{FFT modulo M}
\begin{itemize}
    \item Accetta numeri negativi, bisogna stare attenti ai moduli
    \item (Non me ne prendo responsabilità)
\end{itemize}
\begin{lstlisting}
 #include <bits/stdc++.h>
  using namespace std;

  #define N	100001
  #define L	18	/* L = ceil(log2(N * 2 - 1)) */
  #define N_	(1 << L)
  #define MD	469762049	/* MD = 56 * 2^23 + 1 */

  int *wu[L + 1], *wv[L + 1];

  int power(int a, int k) {
    long long b = a, p = 1;

    while (k) {
      if (k & 1)
        p = p * b % MD;
      b = b * b % MD;
      k >>= 1;
    }
    return p;
  }
  void init() {
    int l, i, u, v;
    u = power(3, (MD - 1) >> L);
    v = power(u, MD - 2);

    for (l = L; l > 0; l--) {
      int n = 1 << (l - 1);

      wu[l] = (int *) malloc(n * sizeof *wu[l]);
      wv[l] = (int *) malloc(n * sizeof *wv[l]);

      wu[l][0] = wv[l][0] = 1;
      for (i = 1; i < n; i++) {
        wu[l][i] = (long long) wu[l][i - 1] * u % MD;
        wv[l][i] = (long long) wv[l][i - 1] * v % MD;
      }

      u = (long long) u * u % MD, v = (long long) v * v % MD;
    }
  }
  void ntt_(int *aa, int l, int inverse) {
    if (l > 0) {
      int n = 1 << l;
      int m = n >> 1;
      int *ww = inverse ? wv[l] : wu[l];
      int i, j;
      ntt_(aa, l - 1, inverse);
      ntt_(aa + m, l - 1, inverse);
      for (i = 0; (j = i + m) < n; i++) {
        int a = aa[i];
        int b = (long long) aa[j] * ww[i] % MD;
        if ((aa[i] = a + b) >= MD)
          aa[i] -= MD;
        if ((aa[j] = a - b) < 0)
          aa[j] += MD;
      }
    }
  }
  void ntt(int *aa, int l, int inverse) {
    int n_ = 1 << l, i, j;
    for (i = 0, j = 1; j < n_; j++) {
      int b;
      int tmp;
      for (b = n_ >> 1; (i ^= b) < b; b >>= 1)
        ;
      if (i < j)
        tmp = aa[i], aa[i] = aa[j], aa[j] = tmp;
    }
    ntt_(aa, l, inverse);
  }
  void mult(int *aa, int n, int *bb, int m, int *out) {
    static int aa_[N_], bb_[N_];
    int l, n_, i, v;
    l = 0;
    while (1 << l <= n - 1 + m - 1)
      l++;
    n_ = 1 << l;
    memcpy(aa_, aa, n * sizeof *aa), memset(aa_ + n, 0, (n_ - n) * sizeof *aa_);
    memcpy(bb_, bb, m * sizeof *bb), memset(bb_ + m, 0, (n_ - m) * sizeof *bb_);
    ntt(aa_, l, 0), ntt(bb_, l,  0);
    for (i = 0; i < n_; i++)
      out[i] = (long long) aa_[i] * bb_[i] % MD;
    ntt(out, l, 1);
    v = power(n_, MD - 2);
    for (i = 0; i < n_; i++)
      out[i] = (long long) out[i] * v % MD;
  }
  int main() {
    static int aa[N], bb[N], out[N_];
    int n, m, i;
    init();
    scanf("%d%d", &n, &m), n++, m++;
    for (i = 0; i < n; i++)
      scanf("%d", &aa[i]);
    for (i = 0; i < m; i++)
      scanf("%d", &bb[i]);
    mult(aa, n, bb, m, out);
    for (i = 0; i < n + m - 1; i++)
      printf("%d ", out[i]);
      printf("\n");
    return 0;
  }
\end{lstlisting}

\subsection{Euclide esteso, ma iterativo}
\begin{lstlisting}
int gcd(int a, int b, int& x, int& y) {
    x = 1, y = 0;
    int x1 = 0, y1 = 1, a1 = a, b1 = b;
    while (b1) {
        int q = a1 / b1;
        tie(x, x1) = make_tuple(x1, x - q * x1);
        tie(y, y1) = make_tuple(y1, y - q * y1);
        tie(a1, b1) = make_tuple(b1, a1 - q * b1);
    }
    return a1;
}
\end{lstlisting}

\subsection{Convex Hull}
\begin{lstlisting}
  struct pt {
      double x, y;
  };

  int orientation(pt a, pt b, pt c) {
      double v = a.x*(b.y-c.y)+b.x*(c.y-a.y)+c.x*(a.y-b.y);
      if (v < 0) return -1; // clockwise
      if (v > 0) return +1; // counter-clockwise
      return 0;
  }
  bool cw(pt a, pt b, pt c, bool include_collinear) {
      int o = orientation(a, b, c);
      return o < 0 || (include_collinear && o == 0);
  }
  bool collinear(pt a, pt b, pt c) { return orientation(a, b, c) == 0; }
  void convex_hull(vector<pt>& a, bool include_collinear = false) {
      pt p0 = *min_element(a.begin(), a.end(), [](pt a, pt b) {
          return make_pair(a.y, a.x) < make_pair(b.y, b.x);
      });
      sort(a.begin(), a.end(), [&p0](const pt& a, const pt& b) {
          int o = orientation(p0, a, b);
          if (o == 0)
              return (p0.x-a.x)*(p0.x-a.x) + (p0.y-a.y)*(p0.y-a.y)
                  < (p0.x-b.x)*(p0.x-b.x) + (p0.y-b.y)*(p0.y-b.y);
          return o < 0;
      });
      if (include_collinear) {
          int i = (int)a.size()-1;
          while (i >= 0 && collinear(p0, a[i], a.back())) i--;
          reverse(a.begin()+i+1, a.end());
      }

      vector<pt> st;
      for (int i = 0; i < (int)a.size(); i++) {
          while (st.size() > 1 && !cw(st[st.size()-2], st.back(), a[i], include_collinear))
              st.pop_back();
          st.push_back(a[i]);
      }
      a = st;
  }
\end{lstlisting}

\subsection{SPFA}
\begin{itemize}
    \item Ricordarsi di aggiungere limite al numero di esecuzioni se possono esserci cicli negativi, altrimenti va all'infinito
\end{itemize}
\begin{lstlisting}
  const int INF = 1000000000;
  vector<vector<pair<int, int>>> adj;

  bool spfa(int s, vector<int>& d) {
      int n = adj.size();
      d.assign(n, INF);
      vector<int> cnt(n, 0);
      vector<bool> inqueue(n, false);
      queue<int> q;
      d[s] = 0;
      q.push(s);
      inqueue[s] = true;
      while (!q.empty()) {
          int v = q.front();
          q.pop();
          inqueue[v] = false;

          for (auto edge : adj[v]) {
              int to = edge.first;
              int len = edge.second;

              if (d[v] + len < d[to]) {
                  d[to] = d[v] + len;
                  if (!inqueue[to]) {
                      q.push(to);
                      inqueue[to] = true;
                      cnt[to]++;
                      if (cnt[to] > n)
                          return false;  // negative cycle
                  }
              }
          }
      }
      return true;
  }
\end{lstlisting}

\subsection{MST $n^2$}
\begin{lstlisting}
  int n;
  vector<vector<int>> adj; // adjacency matrix of graph
  const int INF = 1000000000; // weight INF means there is no edge

  struct Edge {
      int w = INF, to = -1;
  };
  void prim() {
      int total_weight = 0;
      vector<bool> selected(n, false);
      vector<Edge> min_e(n);
      min_e[0].w = 0;
      for (int i=0; i<n; ++i) {
          int v = -1;
          for (int j = 0; j < n; ++j) {
              if (!selected[j] && (v == -1 || min_e[j].w < min_e[v].w))
                  v = j;
          }

          if (min_e[v].w == INF) {
              cout << "No MST!" << endl;
              exit(0);
          }

          selected[v] = true;
          total_weight += min_e[v].w;
          if (min_e[v].to != -1)
              cout << v << " " << min_e[v].to << endl;

          for (int to = 0; to < n; ++to) {
              if (adj[v][to] < min_e[to].w)
                  min_e[to] = {adj[v][to], v};
          }
      }
      cout << total_weight << endl;
}
\end{lstlisting}

\subsection{MST $m\log n$}
\begin{lstlisting}
  vector<int> parent, rank;
  void make_set(int v) {
      parent[v] = v;
      rank[v] = 0;
  }

  int find_set(int v) {
      if (v == parent[v])
          return v;
      return parent[v] = find_set(parent[v]);
  }
  void union_sets(int a, int b) {
      a = find_set(a);
      b = find_set(b);
      if (a != b) {
          if (rank[a] < rank[b])
              swap(a, b);
          parent[b] = a;
          if (rank[a] == rank[b])
              rank[a]++;
      }
  }
  struct Edge {
      int u, v, weight;
      bool operator<(Edge const& other) {
          return weight < other.weight;
      }
  };

  int n;
  vector<Edge> edges;

  int cost = 0;
  vector<Edge> result;
  parent.resize(n);
  rank.resize(n);
  for (int i = 0; i < n; i++)
      make_set(i);

  sort(edges.begin(), edges.end());

  for (Edge e : edges) {
      if (find_set(e.u) != find_set(e.v)) {
          cost += e.weight;
          result.push_back(e);
          union_sets(e.u, e.v);
      }
  }
\end{lstlisting}

\subsection{Matexp}
\begin{lstlisting}
#include <bits/stdc++.h>
using namespace std;

const int N = 3;

const long long M = 1000000007;

void multiply (long long A[N][N], long long B[N][N]){
    long long R[N][N];

    for (int i = 0; i < N; i++){
        for (int j = 0; j < N; j++){
            R[i][j] = 0;
            for (int k = 0; k < N; k++){
                R[i][j] = (R[i][j] + A[i][k] * B[k][j]) % M;
            }
        }
    }

    for (int i = 0; i < N; i++){
        for (int j = 0; j < N; j++){
            A[i][j] = R[i][j];
        }
    }
}

void power_matrix (long long A[N][N], int n){
    long long B[N][N];

    for (int i = 0; i < N; i++){
        for (int j = 0; j < N; j++){
            B[i][j] = A[i][j];
        }
    }

    n = n - 1;
    while (n > 0)
    {
        if (n & 1)
            multiply (A, B);

        multiply (B,B);

        n = n >> 1;  
    }
}

long long solve_recurrence (long long A[N][N], long long B[N][1], int n){
    if (n < N)
        return B[N - 1 - n][0];
    
    power_matrix (A, n - N + 1);
    
    long long result = 0;
    
    for (int i = 0; i < N; i++)
        result = (result + A[0][i] * B[i][0]) % M;
    
    return result;
}

int main ()
{

    long long A[N][N] = {{2, 1, 3}, {1, 0, 0}, {0, 1, 0}};
    long long B[N][1] = {{3}, {2}, {1}};
    
    int n = 5;
    
    long long R_n = solve_recurrence (A, B, n);
    
    cout << "R_" << n << " = " << R_n; 

    return 0;
}
\end{lstlisting}

\subsection{Gauss}
\begin{lstlisting}
const double EPS = 1e-9;
const int INF = 2; // no need for infinity

int gauss (vector < vector<double> > a, vector<double> & ans) {
    int n = (int) a.size();
    int m = (int) a[0].size() - 1;

    vector<int> where (m, -1);
    for (int col=0, row=0; col<m && row<n; ++col) {
        int sel = row;
        for (int i=row; i<n; ++i)
            if (abs (a[i][col]) > abs (a[sel][col]))
                sel = i;
        if (abs (a[sel][col]) < EPS)
            continue;
        for (int i=col; i<=m; ++i)
            swap (a[sel][i], a[row][i]);
        where[col] = row;

        for (int i=0; i<n; ++i)
            if (i != row) {
                double c = a[i][col] / a[row][col];
                for (int j=col; j<=m; ++j)
                    a[i][j] -= a[row][j] * c;
            }
        ++row;
    }

    ans.assign (m, 0);
    for (int i=0; i<m; ++i)
        if (where[i] != -1)
            ans[i] = a[where[i]][m] / a[where[i]][i];
    for (int i=0; i<n; ++i) {
        double sum = 0;
        for (int j=0; j<m; ++j)
            sum += ans[j] * a[i][j];
        if (abs (sum - a[i][m]) > EPS)
            return 0;
    }

    for (int i=0; i<m; ++i)
        if (where[i] == -1)
            return INF;
    return 1;
}
\end{lstlisting}

\end{document}
